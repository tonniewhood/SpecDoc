% Begin Navigation
\subsection{Navigation}

\subitem{
    The Navigation module provides accurate state estimation and sensor selection to support planning and control by running estimators, selecting appropriate sensors, and publishing state estimates. The Navigation module is comprised of two submodules: Sensor Selector and Estimator. Each submodule contributes to robust and accurate state estimation from available sensor inputs.
}

\subsubsection{Interfaces}
\paragraph{Inputs}
\begin{enumerate}
    \item IMU, GPS, barometer, and magnetometer data from all available units
    \item Navigation configuration parameters
\end{enumerate}
\paragraph{Outputs}
\begin{enumerate}
    \item Global position, velocity, attitude, and angular velocity state estimate and uncertainty with timestamp
    \item Logging data for sensor selection and estimation events.
\end{enumerate}
\paragraph{Internal}
\begin{enumerate}
    \item Selected IMU, GPS, barometer, and magnetometer sensor feeds
\end{enumerate}

\subsubsection{Sensor Selector}
\subitem{
    The Sensor Selector submodule evaluates available sensor inputs and using configuration selects the best sensors
    for the current conditions, ensuring robust operation and allowing user input.
}

\begin{enumerate}
% \paragraph{Functional Requirements}
    \item The sensor selector unit shall consume all available sensor feeds defined in 4.4.1.1.1
    \item The sensor selector unit shall consume sensor selection configuration as defined in 4.4.1.1.2
    \item The sensor selector unit shall output selected sensor feeds for the estimator as defined in 4.4.1.3.1
% \textbf{Energy}
    \item The sensor selector shall not consume more than 5\% of the selected processing unit's bandwidth.
% \textbf{Environments}
    \item The sensor selector shall be functional agnostic to the UAV environment.
% \textbf{Safety}
    \item The sensor selector shall expose safe fallbacks sensors if available when preferred sensors are unavailable.
    \item The sensor selector shall use health indicators to avoid selecting faulty sensors.
    \item The sensor selector shall log sensor selection decisions for diagnostics.
% \textbf{Structure}
    \item The sensor selector shall expose priority and fallback choices for sensors based on configuration.
    \item The sensor selector shall only execute upon reception of new sensor data.
% \textbf{Standards and Regulations}
    % \item Follow applicable software development standards.
\end{enumerate}

\subsubsection{Estimator}
\subitem{
    The Estimator submodule fuses selected sensor inputs using sensor fusion algorithms
    to produce the best estimate of the system state.
}

\begin{enumerate}
% \paragraph{Functional Requirements}
    \item The estimator shall provide state estimate outputs and uncertainties as defined in 4.4.1.2.1.
    \item The estimator shall consume selected sensor streams defined in 4.4.1.3.1.
    \item The estimator shall comsume navigation configuration parameters as defined in 4.4.1.1.2.
    \item The estimator shall log raw sensor inputs and state estimates for diagnostics.
% \textbf{Energy}
    \item The estimator shall not consume more than 10\% of the selected processing unit's bandwidth.
% \textbf{Environments}
    \item The estimator shall estimate and compensate for signals with up to 20\% noise.
    \item The estimator shall estimate and correct biases in sensor measurements within 20\%.
    \item The estimator shall expose uncertainties that are accurate to 3 standad deviations.
    \item The estimator shall use the configuration to be congnisant to sensor mounting orientations.
% \textbf{Safety}
    \item The estimator shall log inconsistant inputs and anomalies.
    \item The estimator shall use timestamps to identify and drop stale data.
    \item The estimator shall ensure proper timestamps accompany each output estimate.
% \textbf{Structure}
    \item The estimator shall provide a common interface for inputs of various measurement types.
    \item The estimator shall expose interfaces with data that adheres to at most one body frame coordinate system and one inertial frame coordinate system.
% \textbf{Standards and Regulations}
    % \item Follow applicable software development standards.
\end{enumerate}

% \subsubsection{Functional Requirements}
%     \begin{enumerate}
%         \item Provide timely, bounded-uncertainty state estimates.
%     \end{enumerate}

% \subsubsection{Non-Functional Requirements}
% \textbf{Energy}
% \begin{enumerate}
%     \item Complete estimation within allocated compute budget.
% \end{enumerate}
% \textbf{Environments}
% \begin{enumerate}
%     \item Maintain accuracy across sensor noise and environmental changes.
% \end{enumerate}
% \textbf{Safety}
% \begin{enumerate}
%     \item Provide validity flags and fail-safe estimates upon degraded inputs.
% \end{enumerate}
% \textbf{Structure}
% \begin{enumerate}
%     \item Provide uncertainty metrics and timestamps with each estimate.
% \end{enumerate}
% \textbf{Standards and Regulations}
% \begin{enumerate}
%     \item Follow applicable estimation and data handling standards.
% \end{enumerate}

%%% End Navigation