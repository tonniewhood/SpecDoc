% Begin Planner
\subsection{Planner}

    \subitem{
        The purpose of the Planner is to generate a series of realizable states for the UAV. 
        The planner allows for both autonomous and pilot-directed operation. Inputs from both autonomous
        and pilot-directed modes are evaluated and potentially adjusted to fit safety and feasibility constraints. 
        The inputs are evaluated based on the current UAV state and the current tolerances from configuration parameters.
        The Planner is comprised of three submodules: Waypoint Planner, RC Mixer, and State Select. Each has aids in the overall functionality of the Planner.
    }

    \subsubsection{Interfaces}
    \paragraph{Inputs}
    \begin{enumerate}
        \item The planner requires the mission parameters and UAS tolerances when generating waypoints. These tolerances shall include maximum and minimum airspeeds, maximum g-load, maximum climb and descent rates, maximum altitude, maximum bank angle, and minimum turn radius. Additionally, the planner's configuration parameters shall include an operating mode to determine whether RC inputs or generated waypoints are used to generate the desired states.
        \item Waypoints as defined in 4.1.1.1.2.
        \item RC inputs as defined in 4.1.1.1.3.
        \item UAV state estimates as defined in 4.1.1.3.1.
    \end{enumerate}
    \paragraph{Outputs}
    \begin{enumerate}
        \item Desired States as defined in 4.1.1.3.2.
        \item State mask as defined in 4.1.1.3.3.
        \item Logging data as defined in 4.1.1.3.6.
    \end{enumerate}
    \paragraph{Internal}
    \begin{enumerate}
        \item Planned waypoint states. These inputs are generated by the Waypoint Planner submodule and consumed by the State Select submodule where they'll be multiplexed based on operating mode.
    \end{enumerate}

    \subsubsection{General Requirements}
    \begin{enumerate}
        \item The Planner shall accept the inputs defined in \S4.2.1.1 and produce a time-ordered sequence of desired states while consuming no more than 20\% of the total CPU time allocated to the onboard software system.
        \item For all output states: \subitem{
        \begin{align*}
            V_\text{min} \le v \le V_\text{max}, &\text{ (airspeed) } \\
            |\phi| \le \phi_\text{max}, &\text{ (bank angle) } \\
            |\dot{h}| \le \dot{h}_\text{max}, &\text{ (vertical speed) } \\
            \kappa \le 1/R_\text{min}, &\text{ (turn rate) }\\
            n \le n_\text{max}, &\text{ (load factor) }
        \end{align*}
        Where $V_\text{min}$, $V_\text{max}$, $\phi_\text{max}$, $\dot{h}_\text{max}$, $R_\text{min}$, and $n_\text{max}$ are the UAV tolerances input to the Planner defined in \S4.2.1.1.1.
        }
        \item The Planner shall validate input presence, ranges, units, and timestamps. Invalid inputs shall result in rejection of the current input and preservation of last valid output for $\leq 1$ cycle.
        \item The Planner shall log all input and output data during operation. Additionally, all detected errors will be logged with reason and timestamp.
        \item The Planner shall adhere to all coding standards defined in \S X.X.X.
    \end{enumerate}

    \subsubsection{Waypoint Planner}
    \subitem{
        The Waypoint Planner submodule is responsible for generating feasible waypoint trajectories 
        during autonomous flight. It consumes high level waypoints, and creates a series of states that
        are determined to be safe and feasible based on the current UAV state and configuration parameters.
    }

    \begin{enumerate}
        \item The Waypoint Planner shall include each commanded waypoint as a goal in the output sequence when inclusion does not violate any constraint in \S4.2.2.2, otherwise, it shall mark the waypoint as unreachable and produce the state that best approximates the commanded waypoint while still satisfying all constraints.
        \item The Waypoint Planner shall use no more than 10\% of the total CPU time allocated to the onboard software system.
    \end{enumerate}

    \subsubsection{RC Mixer}
    \subitem{ 
        The RC Mixer submodule is responsible for generating desired states based on pilot RC inputs. These inputs will
        be adjusted to the level defined in the configuration parameters to aid in preserving the safety of the UAV. The 
        adjusted inputs will then be converted to desired states for the UAV to achieve.
    }

    \begin{enumerate}
        \item The RC Mixer shall always produce a valid output regardless of input validity.
        \item The RC Mixer shall converge to zero climb rate and zero turn rate in the absence of valid RC inputs within 2 seconds. 
        \item The RC Mixer shall log all RC inputs and generated outputs with timestamps, any errors detected, and any adjustments made to the inputs to preserve safety.
        \item The RC Mixer shall use no more than 7.5\% of the total CPU time allocated to the onboard software system.
    \end{enumerate}

    \subsubsection{State Select}
    \subitem{ 
        The State Select submodule is responsible for multiplexing between the Waypoint Planner and RC Mixer outputs based on the current operating mode.
        Additionally, the State Select submodule is responsible for determining the state mask passed to the Controller based on the current operating mode.
    }
    \begin{enumerate}
        \item The State Select shall always produce a valid output regardless of input validity.
        \item The State Select shall use no more than 2.5\% of the total CPU time allocated to the onboard software system.
    \end{enumerate}

%%% End Planner