% Begin Planner
\subsection{Planner}

    \subitem{
        The purpose of the Planner is to generate a series of realizable states for the UAV. 
        The planner allows for both autonomous and pilot-directed operation. Inputs from both autonomous
        and pilot-directed modes are evaluated and potentially adjusted to fit safety and feasibility constraints. 
        The inputs are evaluated based on the current UAV state and the current tolerances from configuration parameters.
        The Planner is comprised of three submodules: Waypoint Planner, RC Mixer, and State Select. Each has aids in the overall functionality of the Planner.
    }

    \subsubsection{Interfaces}
    \paragraph{Inputs}
    \begin{enumerate}
        \item Planner configuration parameters.
        \item Goal waypoints comprised of planned global positions with associated deadlines.
        \item Manual RC inputs
        \item State estimates as defined in \S4.2.1.2.1
    \end{enumerate}
    \paragraph{Outputs}
    \begin{enumerate}
        \item A goal state comprised of position, velocity, orientation, and angular velocity.
        \item A goal state mask which determines which state elements are to be considered by the Controller.
        \item Logging data containing any combination of input data, output data, RC input adjustments, waypoint adjustments, and error reports.
    \end{enumerate}
    \paragraph{Internal}
    \begin{enumerate}
        \item Goal waypoints defined in \S4.3.1.1.2 generated by the Waypoint Planner.
        \item Goal waypoints defined in \S4.3.1.1.2 generated by the RC Mixer.
    \end{enumerate}

    \subsubsection{General Requirements}
    \begin{enumerate} 
        \item For all output states: \subitem{
        \begin{align*}
            V_\text{min} \le v \le V_\text{max}, &\text{ (airspeed) } \\
            |\phi| \le \phi_\text{max}, &\text{ (bank angle) } \\
            |\dot{h}| \le \dot{h}_\text{max}, &\text{ (vertical speed) } \\
            \kappa \le 1/R_\text{min}, &\text{ (turn rate) }\\
            n \le n_\text{max}, &\text{ (load factor) }
        \end{align*}
        Where $V_\text{min}$, $V_\text{max}$, $\phi_\text{max}$, $\dot{h}_\text{max}$, $R_\text{min}$, and $n_\text{max}$ shall be tolerances contained in the Planner configuration parameters.
        }
        \item The Planner shall validate the presence of valid inputs as defined in \S4.3.1.1.2 and \S4.3.1.1.3.
        \item The Planner shall log all input and output data during operation.
        \item The Planner shall adhere to all coding standards defined under \S4.1.5.
        \item The Planner shall not consume more than 20\% of the total CPU time allocated to the onboard software system.
    \end{enumerate}

    \subsubsection{Waypoint Planner}
    \subitem{
        The Waypoint Planner submodule is responsible for generating feasible waypoint trajectories 
        during autonomous flight. It consumes high level waypoints, and creates a series of states that
        are determined to be safe and feasible based on the current UAV state and configuration parameters.
    }

    \begin{enumerate}
        \item The Waypoint Planner shall receive configuration parameters defining airframe constraints as defined in \S4.3.2.1.
        \item The Waypoint Planner shall always produce a valid output regardless of input validity.
        \item The Waypoint Planner shall include each commanded waypoint as a goal in the output sequence when inclusion does not violate any constraint in \S4.3.3.1
        \item The Waypoint Planner shall produce outputs that minimize deviation from a commanded waypoint when achieving that commanded waypoint violates any constraint in \S4.3.3.1.
        \item The Waypoint Planner shall log all deviations from commanded waypoints with reasons and timestamps.
        \item The Waypoint Planner shall log all detected faults with reasons and timestamps.
        \item The Waypoint Planner shall use no more than 10\% of the total CPU time allocated to the onboard software system.
    \end{enumerate}

    \subsubsection{RC Mixer}
    \subitem{ 
        The RC Mixer submodule is responsible for generating desired states based on pilot RC inputs. These inputs will
        be adjusted to the level defined in the configuration parameters to aid in preserving the safety of the UAV. The 
        adjusted inputs will then be converted to desired states for the UAV to achieve.
    }

    \begin{enumerate}
        \item The RC Mixer shall receive configuration parameters defining the contents of RC data and their tolerances.
        \item The RC Mixer shall produce goal waypoints that achieve the current RC inputs when those inputs are valid and within tolerances as defined in \S4.3.4.1.
        \item The RC Mixer shall always produce a valid output regardless of input validity.
        \item The RC Mixer shall adjust any RC inputs that violate the tolerances defined in \S4.3.4.1 to bring those inputs within tolerances.
        \item The RC Mixer shall produce goal waypoints that converge to zero climb rate and zero turn rate in the absence of valid RC inputs within 2 seconds. 
        \item The RC Mixer shall log any input adjustments made with reasons and timestamps.
        \item The RC Mixer shall log all detected faults with reasons and timestamps.
        \item The RC Mixer shall use no more than 7.5\% of the total CPU time allocated to the onboard software system.
    \end{enumerate}

    \subsubsection{State Select}
    \subitem{ 
        The State Select submodule is responsible for multiplexing between the Waypoint Planner and RC Mixer outputs based on the current operating mode.
        Additionally, the State Select submodule is responsible for determining the state mask passed to the Controller based on the current operating mode.
    }
    \begin{enumerate}
        \item The State Select shall receive configuration parameters that define the type of state mask to be used.
        \item The State Select shall produce goal states and state masks based on the current operating mode.
        \item The State Select shall always produce a valid output regardless of input validity.
        \item The State Select shall log all detected faults with reasons and timestamps.
        \item The State Select shall use no more than 2.5\% of the total CPU time allocated to the onboard software system.
    \end{enumerate}

%%% End Planner