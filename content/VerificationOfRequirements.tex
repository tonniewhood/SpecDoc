\section{VERIFICATION OF REQUIREMENTS}

% Begin System Verification
\subsection{System Verification}

\subitem{
    % Verification of all System requirements.
}

\subsubsection{Processing}
\subitem{
    % Verification of Processing requirements.
}
\begin{enumerate}
    \item The presence of a central processor shall be verified by inspecting the FCM hardware and datasheet.
    \item The processor speed and hyperperiod shall be verified by executing a representative software load and measuring task execution times to confirm all deadlines within a $2.5$ ms hyperperiod are met.
    \item The use of a real-time operating system for task scheduling shall be verified by inspecting software design documents and source code.
    \item Task isolation shall be verified by intentionally injecting faults into a non-critical task and confirming other tasks continue to operate as expected.
    \item Monitoring of software program state and detection of illegal operations shall be verified by executing a test suite that includes illegal operations and confirming system logs and responses.
    \item Failsafe state activation shall be verified by triggering a fault and demonstrating the system enters a predefined failsafe state (e.g., motor shutdown, error signal).
    \item Coding standard compliance shall be verified using static analysis tools (linters, formatters) and reviewing documentation for adherence to MISRA C:2012, CERT C, POSIX 1003.1, RTCA DO-178C, ARINC 653, and justified deviations.
\end{enumerate}

\subsubsection{Built in HW Capabilities}
\subitem{
    % Verification of Built in HW Capabilities requirements.
}
\begin{enumerate}
    \item The presence of an on-board IMU shall be verified by visual inspection.
    \item The presence of an on-board RC receiver shall be verified by visual inspection.
    \item The presence of an on-board micro-SD card reader shall be verified by visual inspection.
    \item On-board power monitoring shall be verified by measuring output of the power monitoring circuit and comparing to software-reported values.
    \item On-board power regulation shall be verified by applying a range of input voltages and measuring regulated output voltage for tolerance.
\end{enumerate}

\subsubsection{HW Interfaces}
\subitem{
    % Verification of HW Interfaces requirements.
}
\begin{enumerate}
    \item Pins for the I$^2$C interface shall be verified by inspecting the FCM datasheet and PCB.
    \item Pins for the SPI interface shall be verified by inspecting the FCM datasheet and PCB.
    \item Pins for the UART interface shall be verified by inspecting the FCM datasheet and PCB.
    \item Pins for the CAN bus interface shall be verified by inspecting the FCM datasheet and PCB.
    \item Micro SD slot connection to SPI interface shall be verified by visual inspection.
    \item USB A port connection to USB 2.0 interface shall be verified by visual inspection.
    \item Pins for at least 6 PWM channels shall be verified by visual inspection.
    \item Soldering of ports for each interface shall be verified by physical inspection.
    \item Mechanical connection specification compliance shall be verified by reviewing datasheet and measuring connectors for FAA L-823 3.3.3 compliance.
\end{enumerate}

\subsubsection{Hardware Requirements}
\subitem{
    % Verification of Hardware requirements.
}
\begin{enumerate}
    \item Cross-sectional area shall be verified by measuring with calipers and confirming $\leq 412.5$ cm$^2$.
    \item Depth shall be verified by measuring with calipers and confirming $\leq 7.5$ cm.
    \item Mass shall be verified by weighing on a calibrated scale and confirming $\leq 500$ grams.
    \item Altitude operation shall be verified via MIL-STD-810H Method 500.6 low pressure testing.
    \item Temperature operation shall be verified via MIL-STD-810H Methods 501.7 and 502.7.
    \item Humidity operation shall be verified via MIL-STD-810H Method 507.6.
    \item Vibration tolerance shall be verified by mounting on a vibration table and monitoring IMU readings and system functionality.
    \item PCB design standard compliance shall be verified by reviewing PCB design files and documentation for IPC-2221.
    \item RoHS compliance shall be verified by reviewing BOM and component datasheets.
    \item FCC Part 15 compliance shall be verified by testing at an accredited facility or reviewing certification documentation for RF modules.
\end{enumerate}

\subsubsection{Software Requirements}
\subitem{
    % Verification of Software requirements.
}
\begin{enumerate}
    \item GPIO interface shall be verified using a software test as defined by the hardware platform.
    \item I$^2$C interface shall be verified using a software test as defined in UM10204 from NXP Semiconductors.
    \item SPI interface shall be verified using a software test as defined by the hardware platform.
    \item UART interface shall be verified using a software test as defined by the hardware platform.
    \item USB 2.0 interface shall be verified using a software test as defined in the USB 2.0 Specification from USB-IF.
    \item CAN interface shall be verified using a software test as defined in ISO 11898-1:2015.
    \item Micro SD card interface shall be verified using a software test as defined in the SD Physical Layer Simplified Specification v6.00.
    \item ADC channel interface shall be verified using a software test as defined by the hardware platform.
    \item Hardware timer interface shall be verified using a software test as defined by the hardware platform.
    \item Interrupt configuration and handling shall be verified using a software test as defined by the hardware platform.
    \item Device interface via provided hardware interfaces and drivers shall be verified using a software test.
    \item ICD outlining APIs shall be verified by reviewing documentation.
    \item C code compliance shall be verified using static analysis tools for BARR-C:2018 and CERT C.
    \item C++ code compliance shall be verified using static analysis tools for C++ Core Guidelines and CERT C++.
    \item Assembly code compliance shall be verified using static analysis tools for manufacturer coding standards.
    \item Coding standards for other languages shall be verified by reviewing documentation for appropriateness.
    \item Real-time code compliance shall be verified by reviewing code and documentation for POSIX 1003.1, RTCA DO-178C, and ARINC 653.
    \item Documentation and justification of deviations shall be verified by reviewing the Software Design Document.
\end{enumerate}

\subsubsection{Power Management}
\subitem{
    % Verification of Power Management requirements.
}
\begin{enumerate}
    \item Power input voltage range shall be verified by connecting a variable DC power supply and monitoring operational status from $4.75V$ to $5.25V$.
    \item Overcurrent protection shall be verified by increasing current above $5$ amps and confirming activation of protection mechanisms.
    \item Low voltage initialization protection shall be verified by setting power supply below $4.75V$ and confirming device does not initialize, then raising voltage to confirm correct initialization.
    \item SOC monitoring shall be verified by varying input voltage and comparing software-reported voltage to multimeter readings.
    \item Critically low voltage protection shall be verified by lowering input voltage to $4.25V$ and confirming the planner initiates its low-voltage response protocol.
\end{enumerate}

%%% End System Verification

% Begin Planner Verification
\subsection{Planner Verification}

\subitem{
    % Verification of all Planner requirements.
}

\subsubsection{Waypoint Planner}
\subitem{
    % Verification of Waypoint Planner requirements.
}
\begin{enumerate}
    % \paragraph{Functional Requirements}
    \item Verify consumption of waypoints and configuration parameters.
    \item Verify output of planned waypoints.
    % \textbf{Energy}
    \item Verify execution within allocated CPU and power budgets.
    % \textbf{Environments}
    \item Verify performance across expected environmental conditions.
    % \textbf{Safety}
    \item Verify adherence to safety constraints.
    % \textbf{Structure}
    \item Verify modular design and clear interfaces.
    % \textbf{Standards and Regulations}
    \item Verify compliance with applicable software development standards.
\end{enumerate}

\subsubsection{RC Mixer}
\subitem{
    % Verification of RC Mixer requirements.
}
\begin{enumerate}
    % \paragraph{Functional Requirements}
    \item Verify consumption of required items.
    \item Verify output of required items.
    % \textbf{Energy}
    \item Verify execution within allocated CPU and power budgets.
    % \textbf{Environments}
    \item Verify performance across expected environmental conditions.
    % \textbf{Safety}
    \item Verify adherence to safety constraints.
    % \textbf{Structure}
    \item Verify modular design and clear interfaces.
    % \textbf{Standards and Regulations}
    \item Verify compliance with applicable software development standards.
\end{enumerate}

\subsubsection{State Select}
\subitem{
    % Verification of State Select requirements.
}
\begin{enumerate}
    % \paragraph{Functional Requirements}
    \item Verify consumption of required items.
    \item Verify output of required items.
    % \textbf{Energy}
    \item Verify execution within allocated CPU and power budgets.
    % \textbf{Environments}
    \item Verify performance across expected environmental conditions.
    % \textbf{Safety}
    \item Verify adherence to safety constraints.
    % \textbf{Structure}
    \item Verify modular design and clear interfaces.
    % \textbf{Standards and Regulations}
    \item Verify compliance with applicable software development standards.
\end{enumerate}

% \subsubsection{Functional Requirements}
%     \begin{enumerate}
%         \item Verify generation of feasible waypoint trajectories based on input commands and UAV state.
%     \end{enumerate}

% \subsubsection{Non-Functional Requirements}
%     % \textbf{Energy}
%     % \begin{enumerate}
%     %     \item Verify operation within the power budget allocated for the onboard processor.
%     % \end{enumerate}
%     % \textbf{Environments}
%     % \begin{enumerate}
%     %     \item Verify tolerance to expected operating temperatures and vibration levels.
%     % \end{enumerate}
%     % \textbf{Safety}
%     % \begin{enumerate}
%     %         \item Verify that unsafe actuator commands are not issued under fault conditions.
%     % \end{enumerate}
%     % \textbf{Structure}
%     % \begin{enumerate}
%     %     \item Verify exposure of well-defined APIs for submodules.
%     % \end{enumerate}
%     % \textbf{Standards and Regulations}
%     % \begin{enumerate}
%     %     \item Verify compliance with applicable coding and safety guidelines.
%     % \end{enumerate}

%%% End Planner Verification

% Begin Controller Verification
\subsection{Controller Verification}

\subitem{
    % Verification of all Controller requirements.
}

\subsubsection{Preprocessor}
\subitem{
    % Verification of Preprocessor requirements.
}
\begin{enumerate}
% \paragraph{Functional Requirements}
    \item Verify consumption of raw sensor measurements.
    \item Verify output of filtered sensor streams.
% \textbf{Energy}
    \item Verify execution within allocated CPU and power budgets.
% \textbf{Environments}
    \item Verify performance across expected environmental conditions.
% \textbf{Safety}
    \item Verify that filtering does not mask critical fault indicators.
% \textbf{Structure}
    \item Verify modular design and clear interfaces.
% \textbf{Standards and Regulations}
    \item Verify compliance with applicable software development standards.
\end{enumerate}

\subsubsection{PVA Controller}
\subitem{
    % Verification of PVA Controller requirements.
}
\begin{enumerate}
% \paragraph{Functional Requirements}
    \item Verify consumption of waypoints and state estimates.
    \item Verify output of actuator-level setpoints.
% \textbf{Energy}
    \item Verify execution within allocated CPU and power budgets.
% \textbf{Environments}
    \item Verify performance across expected environmental conditions.
% \textbf{Safety}
    \item Verify bounded errors and safe fallback when estimates are invalid.
% \textbf{Structure}
    \item Verify modular design and clear interfaces.
% \textbf{Standards and Regulations}
    \item Verify compliance with applicable software development standards.
\end{enumerate}

\subsubsection{ATL Controller}
\subitem{
    % Verification of ATL Controller requirements.
}
\begin{enumerate}
% \paragraph{Functional Requirements}
    \item Verify consumption of desired attitude and torque references.
    \item Verify output of low-level actuator commands.
% \textbf{Energy}
    \item Verify execution within allocated CPU and power budgets.
% \textbf{Environments}
    \item Verify performance across expected environmental conditions.
% \textbf{Safety}
    \item Verify enforcement of actuator limits and safe modes under faults.
% \textbf{Structure}
    \item Verify modular design and clear interfaces.
% \textbf{Standards and Regulations}
    \item Verify compliance with applicable software development standards.
\end{enumerate}

\subsubsection{Control Distributor}
\subitem{
    % Verification of Control Distributor requirements.
}
\begin{enumerate}
% \paragraph{Functional Requirements}
    \item Verify consumption of controller output commands.
    \item Verify output of hardware actuator signals.
% \textbf{Energy}
    \item Verify execution within allocated CPU and power budgets.
% \textbf{Environments}
    \item Verify performance across expected environmental conditions.
% \textbf{Safety}
    \item Verify prevention of misrouting and provision of diagnostics and traceability.
% \textbf{Structure}
    \item Verify modular design and clear interfaces.
% \textbf{Standards and Regulations}
    \item Verify compliance with applicable software development standards.
\end{enumerate}

% \subsubsection{Functional Requirements}
%     \begin{enumerate}
%         \item Verify computation of actuator setpoints from planned trajectories and state estimates.
%         \item Verify implementation of required control algorithms and scheduling.
%     \end{enumerate}

% \subsubsection{Non-Functional Requirements}
% \textbf{Energy}
% \begin{enumerate}
%     \item Verify execution within the control loop CPU and power budgets.
% \end{enumerate}
% \textbf{Environments}
% \begin{enumerate}
%     \item Verify stability across environmental conditions and sensor noise.
% \end{enumerate}
% \textbf{Safety}
% \begin{enumerate}
%     \item Verify bounded outputs and safe behavior under saturation and faults.
% \end{enumerate}
% \textbf{Structure}
% \begin{enumerate}
%     \item Verify modular controller design with clear scheduling and interfaces.
% \end{enumerate}
% \textbf{Standards and Regulations}
% \begin{enumerate}
%     \item Verify compliance with applicable control and software standards.
% \end{enumerate}

%%% End Controller Verification

% Begin Navigation Verification
\subsection{Navigation Verification}

\subitem{
    % Verification of all Navigation requirements.
}

\subsubsection{Sensor Selector}
\subitem{
    % Verification of Sensor Selector requirements.
}
\begin{enumerate}
% \paragraph{Functional Requirements}
    \item Verify prioritized sensor streams and fallback choices.
    \item Verify consumption of all available sensor feeds and health indicators.
    \item Verify output of selected sensor feeds for estimator.
% \textbf{Energy}
    \item Verify execution within allocated CPU and power budgets.
% \textbf{Environments}
    \item Verify performance across expected environmental conditions.
% \textbf{Safety}
    \item Verify safe fallbacks when preferred sensors are unavailable.
% \textbf{Structure}
    \item Verify modular design and clear interfaces.
% \textbf{Standards and Regulations}
    \item Verify compliance with applicable software development standards.
\end{enumerate}

\subsubsection{Estimator}
\subitem{
    % Verification of Estimator requirements.
}
\begin{enumerate}
% \paragraph{Functional Requirements}
    \item Verify provision of pose, velocity, and other required state variables with uncertainty metrics.
    \item Verify consumption of selected sensor streams and configuration.
    \item Verify output of state estimates with timestamps.
% \textbf{Energy}
    \item Verify execution within allocated CPU and power budgets.
% \textbf{Environments}
    \item Verify performance across expected environmental conditions.
% \textbf{Safety}
    \item Verify safe outputs when inputs are inconsistent or stale.
% \textbf{Structure}
    \item Verify modular design and clear interfaces.
% \textbf{Standards and Regulations}
    \item Verify compliance with applicable software development standards.
\end{enumerate}

% \subsubsection{Functional Requirements}
%     \begin{enumerate}
%         \item Verify timely, bounded-uncertainty state estimates.
%     \end{enumerate}

% \subsubsection{Non-Functional Requirements}
% \textbf{Energy}
% \begin{enumerate}
%     \item Verify estimation within allocated compute budget.
% \end{enumerate}
% \textbf{Environments}
% \begin{enumerate}
%     \item Verify accuracy across sensor noise and environmental changes.
% \end{enumerate}
% \textbf{Safety}
% \begin{enumerate}
%     \item Verify provision of validity flags and fail-safe estimates upon degraded inputs.
% \end{enumerate}
% \textbf{Structure}
% \begin{enumerate}
%     \item Verify provision of uncertainty metrics and timestamps with each estimate.
% \end{enumerate}
% \textbf{Standards and Regulations}
% \begin{enumerate}
%     \item Verify compliance with applicable estimation and data handling standards.
% \end{enumerate}

%%% End Navigation Verification

% Begin Logger Verification
\subsection{Logger Verification}

\subitem{
    % Verification of all Logger requirements.
}

\subsubsection{Storage}
\subitem{
    % Verification of Storage requirements.
}
\begin{enumerate}
% \paragraph{Functional Requirements}
    \item Verify support for retrieval by timestamp and event markers.
    \item Verify persistence of telemetry streams and storage configuration.
    \item Verify retrieval of stored log files or records by time/index.
% \textbf{Energy}
    \item Verify execution within allocated CPU and power budgets.
% \textbf{Environments}
    \item Verify performance across expected environmental conditions.
% \textbf{Safety}
    \item Verify protection of critical logs and prevention of data loss on faults.
% \textbf{Structure}
    \item Verify modular design and clear interfaces.
% \textbf{Standards and Regulations}
    \item Verify compliance with applicable software development standards.
\end{enumerate}

\subsubsection{Transmitt}
\subitem{
    % Verification of Transmitt requirements.
}
\begin{enumerate}
% \paragraph{Functional Requirements}
    \item Verify telemetry prioritization and retransmission.
    \item Verify handling of selected telemetry streams and transmission policy.
    \item Verify transmission of telemetry packets via communications link.
% \textbf{Energy}
    \item Verify execution within allocated CPU and power budgets.
% \textbf{Environments}
    \item Verify performance across expected environmental conditions.
% \textbf{Safety}
    \item Verify prioritization of safety/health telemetry under degraded links.
% \textbf{Structure}
    \item Verify modular design and clear interfaces.
% \textbf{Standards and Regulations}
    \item Verify compliance with applicable software development standards.
\end{enumerate}

% \subsubsection{Functional Requirements}
%     \begin{enumerate}
%         \item Verify preservation of critical telemetry and provision of retrieval interfaces.
%     \end{enumerate}

% \subsubsection{Non-Functional Requirements}
% \textbf{Energy}
% \begin{enumerate}
%     \item Verify logging does not exceed allocated power/CPU budgets.
% \end{enumerate}
% \textbf{Environments}
% \begin{enumerate}
%     \item Verify logging integrity across operating conditions.
% \end{enumerate}
% \textbf{Safety}
% \begin{enumerate}
%     \item Verify logging does not interfere with real-time control operations.
% \end{enumerate}
% \textbf{Structure}
% \begin{enumerate}
%     \item Verify provision of schemas and retention policies for stored data.
% \end{enumerate}
% \textbf{Standards and Regulations}
% \begin{enumerate}
%     \item Verify compliance with applicable data retention and privacy regulations.
% \end{enumerate}

%%% End Logger Verification

% \subsection{Verification Methods}
% 	Possible verification methods include:
% 	\bigskip
	
% 	\begin{enumerate}
% 		\item Inspection:\\

% 	Inspection is a method of verification consisting of investigation, 
% 	without the use of special laboratory appliances or procedures, to 
% 	determine compliance with requirements. Inspection is generally 
% 	nondestructive and includes (but is not limited to) visual examination, 
% 	manipulation, gauging, and measurement.

% 		\item Demonstration:\\

% 	Demonstration is a method of verification that is limited to readily 
% 	observable functional operation to determine compliance with 
% 	requirements. This method shall not require the use of special equipment 
% 	or sophisticated instrumentation.
	
% 		\item Analysis:\\

% 	Analysis is a method of verification, taking the form of the processing of 
% 	accumulated results and conclusions, intended to provide proof that 
% 	verification of a requirement has been accomplished. The analytical 
% 	results may be based on engineering study, compilation or interpretation 
% 	of existing information, similarity to previously verified requirements, 
% 	or derived from lower level examinations, tests, demonstrations, or 
% 	analyses.


% 		\item Direct Test:

% 	Test is a method of verification that employs technical means, including (but not 
% 	limited to) the evaluation of functional characteristics by use of special equipment
% 	or instrumentation, simulation techniques, and the application of established 
% 	principles and procedures to determine compliance with requirements.
			
% 	\end{enumerate}		
	
\newpage
\subsection{Verify Coverage of Stakeholder Requirements}





\begin{table}[h]
\centering
\begin{tabular}{|c|c|C{6cm}|c|c|}
\hline
\textbf{Paragraph Number} & \textbf{Test Type}& 
\textbf{Tester's Name} & \textbf{Pass/Fail} & \textbf{Date} \\
\hline
 & & & & \\
\hline
 & & & & \\
\hline
 & & & & \\
\hline
 & & & & \\
\hline
 & & & & \\
\hline
 & & & & \\
\hline
 & & & & \\
\hline
 & & & & \\
\hline
 & & & & \\
\hline
 & & & & \\
\hline
\end{tabular}
\end{table}

